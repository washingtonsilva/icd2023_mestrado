\documentclass[11pt,]{article}
\usepackage[brazil]{babel}
\usepackage[utf8]{inputenc}
\usepackage[margin = 1in]{geometry}
\newcommand*{\authorfont}{\fontfamily{phv}\selectfont}
\usepackage[]{mathpazo}
\usepackage{abstract}
\renewcommand{\abstractname}{}    % clear the title
\renewcommand{\absnamepos}{empty} % originally center
\newcommand{\blankline}{\quad\pagebreak[2]}

\providecommand{\tightlist}{%
  \setlength{\itemsep}{0pt}\setlength{\parskip}{0pt}} 
\usepackage{longtable,booktabs}

\usepackage{parskip}
\usepackage{titlesec}
\titlespacing\section{0pt}{12pt plus 4pt minus 2pt}{6pt plus 2pt minus 2pt}
\titlespacing\subsection{0pt}{12pt plus 4pt minus 2pt}{6pt plus 2pt minus 2pt}

\titleformat*{\subsubsection}{\normalsize\itshape}

\usepackage{titling}
\setlength{\droptitle}{-.25cm}

%\setlength{\parindent}{0pt}
%\setlength{\parskip}{6pt plus 2pt minus 1pt}
%\setlength{\emergencystretch}{3em}  % prevent overfull lines 

\usepackage[T1]{fontenc}
\usepackage[utf8]{inputenc}

\usepackage{fancyhdr}
\pagestyle{fancy}
\usepackage{lastpage}
\renewcommand{\headrulewidth}{0.3pt}
\renewcommand{\footrulewidth}{0.0pt} 
\lhead{\footnotesize Mestrado Profissional em Administração}
\chead{}
\rhead{\footnotesize Introdução à Ciência dos Dados -- 2022}
\lfoot{}
\cfoot{\small \thepage/\pageref*{LastPage}}
\rfoot{\footnotesize IFMG - Campus Formiga}

\fancypagestyle{firststyle}
{
\renewcommand{\headrulewidth}{0pt}%
   \fancyhf{}
   \fancyfoot[C]{\small \thepage/\pageref*{LastPage}}
}

%\def\labelitemi{--}
%\usepackage{enumitem}
%\setitemize[0]{leftmargin=25pt}
%\setenumerate[0]{leftmargin=25pt}




\makeatletter
\@ifpackageloaded{hyperref}{}{%
\ifxetex
  \usepackage[setpagesize=false, % page size defined by xetex
              unicode=false, % unicode breaks when used with xetex
              xetex]{hyperref}
\else
  \usepackage[unicode=true]{hyperref}
\fi
}
\@ifpackageloaded{color}{
    \PassOptionsToPackage{usenames,dvipsnames}{color}
}{%
    \usepackage[usenames,dvipsnames]{color}
}
\makeatother
\hypersetup{breaklinks=true,
            bookmarks=true,
            pdfauthor={ ()},
             pdfkeywords = {},  
            pdftitle={Introdução à Ciência dos Dados},
            colorlinks=true,
            citecolor=blue,
            urlcolor=blue,
            linkcolor=magenta,
            pdfborder={0 0 0}}
\urlstyle{same}  % don't use monospace font for urls


\setcounter{secnumdepth}{0}





\usepackage{setspace}

\title{Introdução à Ciência dos Dados}
\author{Prof.~Washington Santos da Silva}
\date{2022}


\begin{document}  

		\maketitle
		
	
		\thispagestyle{firststyle}

%	\thispagestyle{empty}


	\noindent \begin{tabular*}{\textwidth}{ @{\extracolsep{\fill}} lr @{\extracolsep{\fill}}}


E-mail: \texttt{\href{mailto:washington.silva@ifmg.edu.br}{\nolinkurl{washington.silva@ifmg.edu.br}}} & Web: \href{http://ead2.ifmg.edu.br/formiga/}{\tt ead2.ifmg.edu.br/formiga/}\\
Aulas: Quinta-Feira - 18:30 - 22:30 & Sala de Aula: Laboratório de
Informática 3 - Bloco B - 2 Andar\\
	&  \\
	\hline
	\end{tabular*}
	
\vspace{2mm}
	


\hypertarget{objetivo-geral}{%
\section{Objetivo Geral}\label{objetivo-geral}}

Capacitar os mestrandos na aplicação de modelos econométricos a
problemas de áreas de finanças. O conteúdo principal do curso
concentra-se na aquisição e estruturação de dados, análise exploratória
de dados, visualização de dados, modelagem econométrica e comunicação
eficaz de resultados, utilizando-se a linguagem R. Trata-se de um curso
do tipo ``hands-on''. O objetivo do curso é levar os mestrandos do zero
à capacidade de trabalhar em um projeto econométrico-computacional
reproduzível usando a linguagem R, analisando um conjunto de dados e
respondendo a questões de interesse.

\hypertarget{objetivos-c}{%
\section{Objetivos c}\label{objetivos-c}}

\begin{enumerate}
\def\labelenumi{\arabic{enumi}.}
\item
  Tornar o mestrando em um usuário da linguagem R para a importação,
  organização, visualização e modelagem econométrica de dados;
\item
  Introduzir o mestrando na aplicação de modelos econométricos na área
  de finanças;
\item
  Tornar o mestrando capacitado para o desenvolvimento de pesquisas
  empíricas reproduzíveis na área de finanças;
\end{enumerate}

\hypertarget{pruxe9-requisitos}{%
\section{Pré-Requisitos}\label{pruxe9-requisitos}}

Não há pré-requisitos formais. Será importante rever conceitos
matemáticos, de estatística básica conforme necessário, sendo importante
que os mestrandos se envolvam no domínio de conceitos básicos
ativamente. Algum letramento computacional também será importante.

\hypertarget{bibliografia}{%
\section{Bibliografia}\label{bibliografia}}

Referência principal

BROOKS, Chris. (2019). \textbf{Introductory Econometrics for Finance}. 4
ed. Cambridge University Press.

Referências Complementares

\hypertarget{linguagem-r}{%
\subsection{Linguagem R}\label{linguagem-r}}

HANCK, Christoph; ARNOLD, Martin; GERBER, Alexander; SCHMELZER, Martin.
\textbf{Introduction to Econometrics with R}. Disponível em:
\url{https://www.econometrics-with-r.org/}

Rodgers, Taylor. \textbf{R Programming in Plain English}. Disponível em:
\url{https://www.rprogramminginplainenglish.com/index.html}

SMAY, Chester; KIM, Albert Y. \textbf{Statistical inference via data
science: A Modern Dive into R and the tidyverse}. Chapman and Hall/CRC,
2019. Disponível em: \url{https://moderndive.com/}

GUERRA, Saulo; OLIVEIRA, Paulo Felipe de; MCDONNEL, Robert; GONZAGA,
Sillas. Ciência de Dados com R - Introdução. Disponível em:
\url{http://sillasgonzaga.com/material/cdr/}

WICKHAM Hadley; GROLEMUND, Garrett. (2017). \textbf{R for data science.
import, tidy, transform, visualize, and model data}. O'Reilly Media,
Inc.~Disponível em: \url{https://r4ds.had.co.nz/}.

\hypertarget{econometria}{%
\subsection{Econometria}\label{econometria}}

WOOLDRIDGE, Jeffrey M. \textbf{Introdução à econometria: uma abordagem
moderna}. São Paulo: Thomson, 2006.

JAMES, G.; WITTEN, D.; HASTIE, T. \& TIBSHRANI, R. \textbf{An
introduction to statistical learning}. 2 edition. New York: Springer,
2021. Disponível em: \url{https://www.statlearning.com/}

HILL, R. Carter; GRIFFITHS, William E.; JUDGE, George G.
\textbf{Econometria}. 3 ed.~ São Paulo: Saraiva, 2010.

MADDALA, G. S. \textbf{Introdução à Econometria}. 3 ed.~Rio de Janeiro:
LTC, 2003.

PERLIN, Marcelo S. (2021). \textbf{Análise de Dados Financeiros e
Econômicos com o R}. self published. Disponível em:
\url{https://www.msperlin.com/adfeR/}.

\hypertarget{poluxedticas-do-curso}{%
\section{Políticas do Curso}\label{poluxedticas-do-curso}}

Use as informações abaixo como referência sobre como o curso será
conduzido. Analise essas informações atentamente antes de entrar em
contato com alguma dúvida.

\hypertarget{distribuiuxe7uxe3o-das-notas}{%
\subsection{Distribuição das Notas}\label{distribuiuxe7uxe3o-das-notas}}

Os pontos referentes à avaliação foram distribuídos da seguinte forma:

\begin{enumerate}
\def\labelenumi{\arabic{enumi})}
\tightlist
\item
  Lista de Exercícios 1 realizada pelo Ambiente Virtual de Aprendizagem
  = 10 pontos
\item
  Lista de Exercícios 2 realizada pelo Ambiente Virtual de Aprendizagem
  = 10 pontos
\item
  Lista de Exercícios 3 realizada pelo Ambiente Virtual de Aprendizagem
  = 10 pontos
\item
  Exame 1 = 35 pontos
\item
  Exame 2 = 35 pontos
\end{enumerate}

Sendo que para aprovação é necessário a obtenção de pelo menos 60
pontos.

\hypertarget{prazos}{%
\subsection{Prazos}\label{prazos}}

Se um evento fora de seu controle fizer com que você perca a data de um
evento ou prazo de entrega, envie-me um e-mail assim que possível para
que eu analise a questão. Como regra, não está planejada a reprogramação
de atividades.

\hypertarget{comunicauxe7uxe3o-e-ambiente-virtual-de-aprendizagem}{%
\subsection{Comunicação e Ambiente Virtual de
Aprendizagem}\label{comunicauxe7uxe3o-e-ambiente-virtual-de-aprendizagem}}

Após a liberação do acesso à sala virtual da disciplina no Ambiente
Virtural de Aprendizagem do IFMG (Moodle), nossa comunicação será
realizada prioritariamente por esta sala virtual.

\hypertarget{poluxedtica-de-honestidade-acaduxeamica}{%
\subsection{Política de Honestidade
Acadêmica}\label{poluxedtica-de-honestidade-acaduxeamica}}

O plágio mina o próprio cerne da missão desta disciplina, que é que cada
mestrando cresça como um acadêmico emergente em pesquisas empíricas na
área de concentração do curso. Sem buscar completar todas as atividades
de forma autonôma, você nunca será capaz de atingir um nível de
proficiência adequado para um mestre. Pior ainda, o plágio reduz a
confiança mútua no trabalho, um componente necessário do empreendimento
científico.

O plágio inclui a cópia no todo ou em parte de trabalhos acadêmicos
publicados anteriormente sem a devida citação, mas também abrange fazer
com que o trabalho de outra pessoa seja seu, independentemente de ter
sido publicado ou não.

Contratar alguém para escrever um ensaio para você é tão plágio quanto
se você copiasse um artigo existente para o seu trabalho. Se houver
evidência consistente de que um mestrando voluntariamente plagiou uma
tarefa, será atribuída a nota zero na tarefa, sendo que o envolvido
poderá ser reprovado na disciplina e/ou receber outras medidas
disciplinares de acordo com as normas institucionais aplicáveis.

\hypertarget{planejamento-do-curso}{%
\section{Planejamento do Curso}\label{planejamento-do-curso}}

Espero que os mestrandos tentem estudar as leituras indicadas,
preferencialmente antes das aulas. Isso não significa apenas ler
superficialmente, mas envolver-se praticamente e criticamente com o
estudo. Especificamente, no que concerne à linguagem R, recomendo
fortemente que busquem fazer todos os exercícios e utilizar efetivamente
a linguagem. É necessário destacar que trata-se de um planejamento e,
como todo planejamento, está sujeito a alterações.

\hypertarget{aula-1---1703}{%
\subsection{Aula 1 - 17/03}\label{aula-1---1703}}

\begin{itemize}
\tightlist
\item
  Apresentação da Disciplina
\item
  Introdução à Linguagem R
\item
  Brooks (2019): Chapter 1
\end{itemize}

\hypertarget{aula-2---2403}{%
\subsection{Aula 2 - 24/03}\label{aula-2---2403}}

\begin{itemize}
\tightlist
\item
  Introdução à Linguagem R
\end{itemize}

\hypertarget{aula-3---3103}{%
\subsection{Aula 3 - 31/03}\label{aula-3---3103}}

\begin{itemize}
\tightlist
\item
  Fundamentos de Estatística
\item
  Tipos de Dados
\item
  Brooks (2019): Chap. 2
\end{itemize}

\hypertarget{aula-4---0704}{%
\subsection{Aula 4 - 07/04}\label{aula-4---0704}}

\begin{itemize}
\tightlist
\item
  Fundamentos de Estatística
\item
  Tipos de Dados
\item
  Brooks (2019): Chap. 2
\end{itemize}

\hypertarget{lista-de-exercuxedcios-1}{%
\subsection{Lista de Exercícios 1}\label{lista-de-exercuxedcios-1}}

Entrega via Moodle até 15/04

\hypertarget{aula-5---2804}{%
\subsection{Aula 5 - 28/04}\label{aula-5---2804}}

\begin{itemize}
\tightlist
\item
  Modelos de Regressão
\item
  Brooks (2019): Chaps. 3, 4.
\end{itemize}

\hypertarget{aula-6---0505}{%
\subsection{Aula 6 - 05/05}\label{aula-6---0505}}

\begin{itemize}
\tightlist
\item
  Modelos de Regressão
\item
  Brooks (2019): Chaps 3, 5
\end{itemize}

\hypertarget{aula-7---1205}{%
\subsection{Aula 7 - 12/05}\label{aula-7---1205}}

\begin{itemize}
\tightlist
\item
  Introdução a Modelos para Séries Temporais
\item
  Brooks (2019): Chap. 6
\end{itemize}

\hypertarget{lista-de-exercuxedcios-2}{%
\subsection{Lista de Exercícios 2}\label{lista-de-exercuxedcios-2}}

Entrega via Moodle até 19/05

\hypertarget{aula-8---1905}{%
\subsection{Aula 8 - 19/05}\label{aula-8---1905}}

\begin{itemize}
\tightlist
\item
  Exame 1
\end{itemize}

\hypertarget{aula-9---2605}{%
\subsection{Aula 9 - 26/05}\label{aula-9---2605}}

\begin{itemize}
\tightlist
\item
  Modelos para Dados em Painel
\item
  Brooks (2019): Chap. 11
\end{itemize}

\hypertarget{aula-10---0206}{%
\subsection{Aula 10 - 02/06}\label{aula-10---0206}}

\begin{itemize}
\tightlist
\item
  Modelos para Dados em Painel
\item
  Brooks (2019): Chap. 11
\end{itemize}

\hypertarget{aula-11---0906}{%
\subsection{Aula 11 - 09/06}\label{aula-11---0906}}

\begin{itemize}
\tightlist
\item
  Modelos para Variáveis Dependentes Limitadas
\item
  Brooks (2019): Chap. 12
\end{itemize}

\hypertarget{lista-de-exercuxedcios-3}{%
\subsection{Lista de Exercícios 3}\label{lista-de-exercuxedcios-3}}

Entrega via Moodle até 16/06

\hypertarget{aula-12---2306}{%
\subsection{Aula 12 - 23/06}\label{aula-12---2306}}

\begin{itemize}
\tightlist
\item
  Modelos para Variáveis Dependentes Limitadas
\item
  Brooks (2019): Chap. 12
\item
  Exame 2
\end{itemize}




\end{document}

\makeatletter
\def\@maketitle{%
  \newpage
%  \null
%  \vskip 2em%
%  \begin{center}%
  \let \footnote \thanks
    {\fontsize{18}{20}\selectfont\raggedright  \setlength{\parindent}{0pt} \@title \par}%
}
%\fi
\makeatother
